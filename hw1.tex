\documentclass[11pt,letterpaper,boxed]{pset}

\usepackage[margin=0.75in]{geometry}
\usepackage{ulem}

\name{Name: \rule{2.5cm}{0.15mm}}
\assignment{Box \# \rule{1.5cm}{0.15mm}}
\class{MATH060 HW1}

\begin{document}

    \problemlist{MATH060 HW1}
    \begin{center}
        2.1: 7, 18, 19, 22, 33, 40 \\
        2.2: 14
    \end{center}
    
    %------------------------- 2.1.7 -----------------------
    
    \begin{problem}[2.1 \#7]
    Find the domain and range of each of the functions given in Exercises 3–7.
    \[f(x,y)=(x+y, \frac{1}{y-1}, x^2+y^2)\]
    \end{problem}
    \newpage
    
    %------------------------- 2.1.18 -----------------------
    
    \begin{problem}[2.1 \#18]
    In each of Exercises 14–23, (a) determine several level curves of the given function f (make sure to indicate the height c of each curve); (b) use the information obtained in part (a) to sketch the graph of f.
    \[f(x,y)=4x^2+9y^2\]
    \end{problem}
    \newpage
    
    %------------------------- 2.1.19 -----------------------
    
    \begin{problem}[2.1 \#19]
    In each of Exercises 14–23, (a) determine several level curves of the given function f (make sure to indicate the height c of each curve); (b) use the information obtained in part (a) to sketch the graph of f.
    \[f(x,y)=xy\]
    \end{problem}
    \newpage
    
    %------------------------- 2.1.22 -----------------------
    
    \begin{problem}[2.1 \#22]
    In each of Exercises 14–23, (a) determine several level curves of the given function f (make sure to indicate the height c of each curve); (b) use the information obtained in part (a) to sketch the graph of f.
    \[f(x,y)=3−2x−y\]
    \end{problem}
    \newpage
    
    %------------------------- 2.1.33 -----------------------
    
    \begin{problem}[2.1 \#33]
    In Exercises 32–36, describe the graph of g(x, y, z) by computing some level surfaces. (If you prefer, use a computer to assist you.)
    \[g(x,y,z)=x^2+y^2-z\]
    \end{problem}
    \newpage
    
    %------------------------- 2.1.40 -----------------------
    
    \begin{problem}[2.1 \#40]
    Sketch or describe the surfaces in $\R^3$ determined by the equations in Exercises 40–46.
    \[z = \frac{x^2}{4} - y^2\]
    \end{problem}
    \newpage
    
    %------------------------- 2.2.14 -----------------------
    
    \begin{problem}[2.2 \#14]
    Evaluate the limits in Exercises 7–21, or explain why the limit fails to exist.
    \[\lim_{(x,y)\to (0,0)} \frac{xy}{x^2+y^2}\]
    \end{problem}
    \newpage
    
\end{document}